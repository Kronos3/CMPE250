\documentclass[twoside,11pt]{article}
%%%%% PACKAGES %%%%%%
\usepackage[english]{babel}
\usepackage{color}
\usepackage[margin=1in]{geometry}
\usepackage{sectsty}
\usepackage{enumitem}
\usepackage{array,mathtools}
\usepackage[makeroom]{cancel}
\newcommand*{\carry}[1][1]{\overset{#1}}
\newcolumntype{B}[1]{r*{#1}{@{\,}r}}

\newcommand\assignmentNumber{1}
\newcommand\studentName{Andrei Tumbar}
\sectionfont{\fontsize{14}{16}\selectfont}

\begin{document}

\title{\vspace{-3cm}Homework \assignmentNumber\\\studentName\vspace{-1.4cm}}
\maketitle

\section{Convert each of the following decimal integers into its natural binary equivalent.}
\begin{enumerate}[label=\alph*)]
\large\item $13 = 8 + 4 + 1 = 2^3 + 2^2 + 2^0 = 1101$
\large\item $44 = 1(32) + 0(16) + 1(8) + 1(4) + 0(2) + 0(1) = 101100$
\end{enumerate}

\section{Convert the following natural binary integer into its decimal equivalent: $1010110$.}
\large\begin{equation*}
1010110 = 2^6 + 2^4 + 2^2 + 2^1 = 64 + 16 + 4 + 2 = 86
\end{equation*}

\section{Perform the following binary addition: $10100 + 00111$.}

\Large\begin{equation*}
\begin{array}{B1}
         & 1 \carry0 1 0 0 \\
    {} + & 0 0 1 1 1 \\\hline
         & 1 1 0 1 1
\end{array}
\end{equation*}

\section{Perform the following hexadecimal addition: 0x66 $+$ 0x75.}

\Large\begin{equation*}
\begin{array}{B2}
               & 6 & 6 \\
    {} +       & 7 & 5 \\\hline
    \text{0x}  & D & B
\end{array}
\end{equation*}

\section{Suppose that P = 0x1234, and Q = 0x ABEF. In 16-bit hexadecimal arithmetic, calculate the value of the following expressions.}

\noindent\begin{minipage}{.5\linewidth}\Large\begin{equation*}
\begin{array}{B5}
               & 1 & \carry{2} & \carry{3} & 4 \\
    {} +       & A & B         &        E  & F \\\hline
    \text{0x}  & B & D         &        2  & 3
\end{array}
\end{equation*}\end{minipage}
\begin{minipage}{.5\linewidth}\Large\begin{equation*}
\begin{array}{B5}
             & A & B & E & F \\
        {} - & 1 & 2 & 3 & 4 \\\hline
             & 9 & 9 & B & B \\\hline
           - & 9 & 9 & B & B
\end{array}
\end{equation*}\end{minipage}
\pagebreak
\section{Compute the results of the following decimal arithmetic operations using 8-bit two’s complement arithmetic. Also, indicate whether or not arithmetic overflow occurs.}

\Large\begin{equation*}
\begin{array}{B1}
         & 0 0 \carry{0} \carry{1} 1 \carry{0} \carry{0} 1 \\
    {} + & 0 0 0 0 1 0 1 1 \\\hline
         & 0 0 1 0 0 1 0 0
\end{array}
\hspace{1cm}\text{No overflow}
\end{equation*}

\Large\begin{equation*}
\begin{array}{B1}
         & 1 1 0 0 0 0 1 0 \\
    {} - & 0 0 1 1 0 0 0 1 \\\hline
         & \carry{1} 1 0 \carry{0} \carry{0} \carry{0} 1 0 \\
    {} + & 1 1 0 0 1 1 1 1 \\\hline
         & 1 0 0 1 0 0 0 1
\end{array}
\hspace{1cm}\text{No Overflow}
\end{equation*}


\end{document}
