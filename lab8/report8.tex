%***********************************************************************
% Lab Report Template
% R. W. Melton
% May 19, 2019
% August 24, 2020
%***********************************************************************
%Modify the following macros to set document property values
%used for cover sheet (title page) and page header
\newcommand{\CourseNumber}{CMPE-250}
\newcommand{\CourseName}{Assembly and Embedded Programming}
\newcommand{\SemesterName}{Fall 2020}
\newcommand{\SemesterCode}{20201}
\newcommand{\LabExNum}{8}
\newcommand{\LabExTitle}{Multiprecision Arithmetic}
\newcommand{\StudentName}{Andrei Tumbar}
\newcommand{\DateSubmit}{Submitted: 10-27-20}
\newcommand{\LabSection}{5}
\newcommand{\LabInstructor}{Gordon Werner}
\newcommand{\TAa}{Tianran Cui}
\newcommand{\TAb}{Anthony Bacchetta}
\newcommand{\LectureSection}{1}
\newcommand{\LectureInstructor}{Melton}
%End macros for document property values
%***********************************************************************
\title{Lab Ex. \LabExNum\ Report}
\author{\StudentName}
\date{\DateSubmit}
\makeatletter %make \title, \author, and \date availabile with \@
\newcommand{\FontSize}{12}
\newcommand{\FontUnit}{pt}
\newcommand{\HeadSize}{\dimexpr \FontSize\FontUnit + 2pt \relax}
\documentclass[\FontSize\FontUnit,letterpaper,oneside]{article}
\usepackage[twoside=false,margin=1in]{geometry}
\usepackage[utf8]{inputenc}
\usepackage[USenglish]{babel}
\usepackage{graphicx}
\usepackage[normalem]{ulem}
\usepackage{newtxtext, newtxmath}
%If newtx package had to be installed in multiuser environment
%regular user may have to run updmap command to avoid following error
%FATAL:  ``PK font ts1-qtmr could not be created.'' in miktex-makepk
%Alternatively, uncomment the following line
%\pdfmapfile{=pdftex35.map
\usepackage{booktabs}
\usepackage{enumitem}
\usepackage{nameref}
\usepackage{hhline}
\usepackage{fontspec,kantlipsum}
\usepackage[T1]{fontenc}
\usepackage{caption}
\captionsetup[table]{skip=10pt}
\usepackage[pdfborder={0 0 0},plainpages=false,pdfpagelabels]{hyperref}
\def\code#1{\texttt{#1}}
%If hyperref generates errors on first build, rebuild.            
\hypersetup{pdfauthor={\@author},
            pdftitle={\@title},
            pdfsubject={\CourseNumber\ \SemesterCode},
            %pdfkeywords={},
            %pdfproducer={Latex with hyperref, or other system},
            %pdfcreator={pdflatex, or other tool}
            urlcolor=none}
\setlength{\topsep}{\z@}
\setlength{\partopsep}{\z@}
\setlength{\itemsep}{\z@}
\setlength{\parindent}{\z@}
\setlength{\parskip}{\FontSize\FontUnit plus 2pt minus 1pt}
\setlength{\baselineskip}{\dimexpr \FontSize\FontUnit + 2pt \relax}
\renewcommand \baselinestretch{1}
\makeatletter
  \renewcommand \section{
    \@startsection{section}{1}{\z@}
      %Before 2 lines, accounting for normal \parskip
      {\dimexpr \FontSize\FontUnit * 2 - \parskip \relax plus 0pt minus 0pt}
      %After 1 line, accounting for normal \parskip
      {0.1pt plus 2pt minus 1pt} %nonzero amount to get normal \parskip
      {\normalfont\normalsize\bfseries}} 
  \renewcommand \subsection{
    \@startsection{paragraph}{2}{\z@}
      %Before 1 lines, accounting for normal \parskip
      {0.1pt plus 2pt minus 1pt}
      %After 0.5 em on same line as heading
      {-0.5em} 
      {\normalfont\normalsize\bfseries}} 
  \renewcommand \subsubsection{
    \@startsection{paragraph}{3}{\z@}
      %Before 1 line, accounting for normal \parskip
      {0.1pt plus 2pt minus 1pt}
      %After 0.5 em on same line as heading
      {-0.5em} 
      {\normalfont\normalsize\uline}} 
  \renewcommand \paragraph{
      \@startsection{paragraph}{4}{\z@}
      %Before 1 line, accounting for normal \parskip
      {0.1pt plus 2pt minus 1pt}
      %After 0.5 em on same line as heading
      {-0.5em} 
      {\normalfont\normalsize}} 
\makeatother
\pagenumbering{arabic}
\headheight=\HeadSize
\usepackage{fancyhdr}
\renewcommand{\headrulewidth}{0pt}
\renewcommand{\footrulewidth}{0pt}
\makeatletter %make \title, \author, and \date availabile with \@
\pagestyle{fancy}
\fancyhead{} %clear all header fields
\fancyhead[L]{\small \CourseNumber\ \SemesterCode \@author:  \@title}
\fancyhead[R]{\small Page \thepage\ of \pageref*{LastPage}}
\fancyfoot{} %clear all footer fields
\fancypagestyle{plain}{
  \renewcommand{\headrulewidth}{0pt}
  \renewcommand{\footrulewidth}{0pt}
  \fancyhf{} %clear header and footer fields
  \fancyfoot[C]{\small \CourseNumber\ \SemesterCode\ \@author:  \@title:  
    Page \thepage\ of \pageref*{LastPage}}
}
%May require second build to get correct page numbers.   


\usepackage{xcolor}
\usepackage{listings}

\definecolor{mGreen}{rgb}{0,0.6,0}
\definecolor{mGray}{rgb}{0.5,0.5,0.5}
\definecolor{mPurple}{rgb}{0.58,0,0.82}
\definecolor{backgroundColour}{rgb}{0.95,0.95,0.92}

\lstdefinestyle{CStyle}{
    commentstyle=\color{mGreen},
    keywordstyle=\color{magenta},
    numberstyle=\tiny\color{mGray},
    stringstyle=\color{mPurple},
    basicstyle=\footnotesize,
    breakatwhitespace=false,         
    breaklines=true,                 
    captionpos=b,                    
    keepspaces=true,                 
    numbers=left,                    
    numbersep=5pt,                  
    showspaces=false,                
    showstringspaces=false,
    showtabs=false,                  
    tabsize=2,
    language=C
}

         
\begin{document}
\raggedbottom
\widowpenalties 1 10000
\lefthyphenmin=4
\righthyphenmin=4
\setlist{nolistsep}
%***********************************************************************
%Title page is automatically generated from macros at top of file
\pagenumbering{roman}
\begin{titlepage}
  %No space before paragraph at top of page
  %\vspace{\dimexpr-2\parsep-2\parskip\relax}
  %1.5 in before center (list) at top of page
  \vspace*{\dimexpr 1.5in - \topsep - \partopsep - \topskip - \parskip \relax}
  \begin{center}
    \textbf{\large\CourseNumber\ \CourseName\linebreak
      \linebreak
      Laboratory Exercise \LabExNum\linebreak
      \linebreak
      \LabExTitle}
  \end{center}
  \vspace*{\dimexpr 1.5in - \topsep - \partopsep - \topskip \relax}
  \par By submitting this report, I attest that its contents are wholly 
    my individual writing about this exercise and that they reflect 
    the submitted code.  I further acknowledge that permitted 
    collaboration for this exercise consists only of discussions of 
    concepts with course staff and fellow students.  Other than code 
    provided by the instructor for this exercise, all code was 
    developed by me.
  \null
  \vspace*{4\parskip}
  \hspace*{3.25in}\begin{tabular}[t]
    {@{\hskip0pt}r    %Specification
     @{\hskip1em}l    %Value
     @{\hskip0pt}}
    \toprule[1pt]
    \multicolumn{2}{l}{\StudentName}\\
    \multicolumn{2}{l}{\DateSubmit}\\
    \\
    Lab Section:&\LabSection\\
    Instructor:&\LabInstructor\\
    TA:&\TAa\\
    &\TAb\\
    \\
    Lecture Section:&\LectureSection\\
    Lecture Instructor:&\LectureInstructor
  \end{tabular}
\end{titlepage}
\pagenumbering{arabic}
\thispagestyle{plain}
%***********************************************************************
%Report body begins here

\section*{Demonstration}

A screen capture of the terminal output of the final program was taken. The screen capture was taken after the final demonstration was performed.

\begin{figure}[h!]
	\centering
	\includegraphics{capture}
	\caption{Terminal screen capture}
	\label{fig:capture1}
\end{figure}

The first operation tests that basic adding functionality works with 16-byte integers. This test is an important baseline to verify that the numbers were stored in little endian format. The next addition tests both the case-insensitive support along with the carry bit from the first word to the second. The next two cases also test that the carry bit is functional between each of the word boundaries.

The next two inputs will nexts that is a carry on the final bit is set, the program will indicate that the addition operation overflowed.

Finally the last two addition operations test whether invalid input will properly re-prompt the user for both the addend and augend.

\section*{Subroutine addresses}

\begin{table}[h!]
\begin{center}
\caption{Code section offsets and endings.}
\label{tab:q1}
\begin{tabular}{ |l||c|c| }
\hline
Subroutine & Address & Size (bytes) \\\hhline{|=||=|=|}
main & \code{0x00000410} & \code{140} \\\hline 
AddIntMultiU & \code{0x0000049d} & \code{56} \\\hline 
GetHexIntMulti & \code{0x000004d5} & \code{116} \\\hline
PutHexIntMulti & \code{0x00000549} & \code{26} \\\hline
\end{tabular}
\end{center}
\end{table}


\label{LastPage}
\end{document}
